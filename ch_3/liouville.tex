\documentclass{article}

\usepackage[utf8]{inputenc}
\usepackage[english]{babel}
\usepackage{amsthm} %lets us use \begin{proof}
\usepackage{amssymb} %gives us the character \varnothing
\usepackage{xcolor}
\usepackage{float}
\usepackage{braket}
\usepackage{multirow}
\usepackage{array}
\usepackage{mathtools}

\title{Chapter 2: Quantum Dynamics in Hilbert Space} % Title of the assignment

\begin{document}

\maketitle

\section{Overview}

Chapter 2 introduces the reader to the Time-Dependent Schr{\"o}dinger equation (TDSE)
and defines the time propagator operator ($U(t,t_0)$). The two level system is
provided as an example of solving the TDSE and the corresponding time propagator operator
is determined. The interaction picture is introduced to simplify the
problem from the Schr{\"o}dinger picture and various properties are shown. Lastly,
the Greens function within the frequency domain is used to model the proposed
Wigner and Weisskopy of irreversible decay.

\section*{Time-Evolution Operator with Time-Dependent Hamiltonian}

Within the Schr{\"o}dinger picture, the TD wavefunction is given by
\begin{equation}
  \psi(\mathbf{x},t) = \langle \mathbf{x}|\psi(t)\rangle
\end{equation}

and the TDSE will be defined as
\begin{equation}
  \frac{\partial|\psi(t)\rangle}{\partial t} = -\frac{i}{\hbar}H|\psi(t)\rangle
  \label{eqn:tdse}
\end{equation}

and solutions to Eqn \eqref{eqn:tdse} are the eigenvalues $E_n$ with corresponding
eigenvectors $|\phi_n(t)\rangle$. These eigenvections satisfy the completness
condition $\sum_n|\phi_n\rangle\langle\phi_n|$. We can expand the wavefunction
$\psi(t)$ within this basis,
\begin{equation}
  |\psi(t)\rangle = \sum_n|\phi_n\rangle\langle\phi_n|
  \label{eqn:basis_expan}
\end{equation}


\end{document}
